\documentclass{article}

\usepackage[margin=1.5in]{geometry}
\usepackage[T1]{fontenc}
\usepackage{mathptmx}
\usepackage{amsmath,amsthm,amssymb,hyperref,tikz,tikz-cd,hyperref,bold-extra}
\usepackage{physics}
\usepackage{upgreek}
\usepackage{halloweenmath}
\usepackage[dvipsnames]{xcolor}

%some useful shortcuts
\newcommand{\R}{\mathbb{R}}  
\newcommand{\Z}{\mathbb{Z}}
\newcommand{\N}{\mathbb{N}}
\newcommand{\Q}{\mathbb{Q}}
\newcommand{\C}{\mathbb{C}}
\newcommand{\Log}{\text{Log}\:}
\newcommand{\im}{\text{im}\:}
\newcommand{\coker}{\text{coker}\:}
\newcommand{\cat}[1]{\textsf{#1}}

\title{Solutions to Emily Rhiel's \textit{Category Theory in Context}}
\author{by Maxine}
\date{\today}

\makeatletter
\newcommand{\tpmod}[1]{{\@displayfalse\pmod{#1}}}
\makeatother
%no idea what this does

\newenvironment{theorem}[2][Theorem]{\begin{trivlist}
\item[\hskip \labelsep {\bfseries #1}\hskip \labelsep {\bfseries #2.}]}{\end{trivlist}}
\newenvironment{lemma}[2][Lemma]{\begin{trivlist}
\item[\hskip \labelsep {\bfseries #1}\hskip \labelsep {\bfseries #2.}]}{\end{trivlist}}
\newenvironment{claim}[2][Claim]{\begin{trivlist}
\item[\hskip \labelsep {\bfseries #1}\hskip \labelsep {\bfseries #2.}]}{\end{trivlist}}
\newenvironment{problem}[2][Problem]{\begin{trivlist}
\item[\hskip \labelsep {\bfseries #1}\hskip \labelsep {\bfseries #2.}]}{\end{trivlist}}
\newenvironment{proposition}[2][Proposition]{\begin{trivlist}
\item[\hskip \labelsep {\bfseries #1}\hskip \labelsep {\bfseries #2.}]}{\end{trivlist}}
\newenvironment{corollary}[2][Corollary]{\begin{trivlist}
\item[\hskip \labelsep {\bfseries #1}\hskip \labelsep {\bfseries #2.}]}{\end{trivlist}}

\newenvironment{solution}{\begin{proof}[Solution]}{\end{proof}}

\begin{document}
\maketitle

\large 
\vspace{0.05in}
\noindent 
Note: Throughout I use "morphism," "map," and "arrow" interchangeably. \\ \\
\textsc{Exercise} 1.1.i. (i) 
\begin{solution}
    Suppose that $f: x \to y$ is a morphism and $g,h: y \rightrightarrows x$ are such that $fg= 1_x = fh$ and $gf = 1_y = hf$. Then
    \[
        g = g 1_x = g (fh) = (gf) h = 1_y h = h
    \]
\end{solution}
\noindent (ii) Consider a morphism $f: x \to y$. Show that if there exists a pair of morphisms $g,h: y \rightrightarrows x$ so that $gf = 1_x$ and $fh = 1_y$, then $g=h$ and $f$ is an isomorphism. \\ \\
\textsc{Exercise} 1.1.ii. 
Let \cat{C} be a category. Show that the collection of isomorphisms in \cat{C} defines a subcategory, the maximal groupoid inside \cat{C}. \\ \\
\textsc{Exercise} 1.1.iii.
\begin{solution}
    (i) Verifying the category axioms for $c/C$ amounts to showing the diagrams below commute, and the equations in \eqref{eqn:3} hold.
    \begin{equation} \label{eqn:1}
        \begin{tikzcd}
            & c \arrow[dl] \arrow[dr]\\
            x \arrow[rr, "\text{id}_x"] && x
        \end{tikzcd}
    \end{equation}
    \begin{equation}\label{eqn:2}
        \begin{tikzcd}
            & c \arrow[dr]\arrow[d]\arrow[dl]\\
            x  \arrow[r] & y \arrow[r]  & z 
        \end{tikzcd}
    \end{equation}
    \begin{equation}\label{eqn:3}
        \begin{tikzcd}
            & c \arrow[dr]\arrow[d]\arrow[dl]\\
            x  \arrow[r] & y \arrow[r, "\text{id}_y"]  & y 
        \end{tikzcd}
        =
        \begin{tikzcd}
            & c \arrow[dr]\arrow[dl]\\
            x \arrow[rr]  && y 
        \end{tikzcd}
        =
        \begin{tikzcd}
            & c \arrow[dr]\arrow[d]\arrow[dl]\\
            x  \arrow[r, "\text{id}_x"] & x \arrow[r]  & y 
        \end{tikzcd}
    \end{equation}
    These properties are inherited by $c/C$ from $C$. 
    %?: examples of this\\ \\
    (ii) This is similar to (i), but with the following diagrams instead. 
    \begin{equation*}
        \begin{tikzcd}
            x \arrow[dr, "f"] \arrow[rr, "\text{id}_x"] && x \arrow[dl, "f"] \\
            & c
        \end{tikzcd}
    \end{equation*}
    \begin{equation*}
        \begin{tikzcd}
                    x \arrow[dr] \arrow[r] & y \arrow[d] \arrow[r] & z \arrow[dl] \\
            & c 
        \end{tikzcd}
    \end{equation*}
    \begin{equation*}
        \begin{tikzcd}
                    x \arrow[dr] \arrow[r] & y \arrow[r, "\text{id}_y"] \arrow[d] & y \arrow[dl] \\
            & c 
        \end{tikzcd}
        =
        \begin{tikzcd}
        x  \arrow[rr] \arrow[dr]&& y \arrow[dl]\\ 
            & c 
        \end{tikzcd}
        =
        \begin{tikzcd}
                    x \arrow[dr] \arrow[r, "\text{id}_x"] & x \arrow[r] \arrow[d] & y \arrow[dl] \\
            & c 
        \end{tikzcd}
    \end{equation*}
    Topological example: Deck transformations.
    \end{solution}
\noindent \textsc{Exercise} 1.2.i. Show that $\cat{C}/c \cong (c/(\cat{C}^{\text{op}}))^{\text{op}}$. Defining $\cat{C}/c$ to be $(c/(\cat{C}^{\text{op}}))^{\text{op}}$, deduce Exercise 1.1.iii(ii) from Exercise 1.1.iii(i).\\ \\
\textsc{Exercise} 1.2.ii. (i) Show that a morphism $f: x\to y$ is a split epimorphism in a category \cat{C} if and only if for all $c \in \cat{C}$, post-composition $f_*: \cat{C}(c,x) \to \cat{C}(c,y)$ defines a surjective function. \\
(ii) Argue by duality that $f$ is a split monomorphism if and only if for all $c \in \cat{C}$, pre-composition $f^*: \cat{C}(y,c) \to \cat{C}(x,c)$ is a surjective function. \\ \\
\textsc{Exercise} 1.2.iii. Prove Lemma 1.2.11 by proving either (i) or (i') and either (ii) or (ii'), then arguing by duality. Conclude that the monomorphisms in any category define a subcategory of that category and dually that the epimorphisms also define a subcategory. \\ \\
\textsc{Exercise} 1.2.iv. 
\begin{solution}
    The monomorphisms in \cat{Field} are precisely the injections. We know injections in a concrete category are always monomorphisms, and field morphisms are always injective. This comes from basic ring theory: the only ideals of a field $F$ are $(1) = F$ and $(0)$, and the kernel of a ring homomorphism is an ideal (forcing the morphism to have either trivial kernel, or send everything to zero). In the (reasonable) case we require field morphisms to send 1 to 1, then the kernel must be trivial.
\end{solution}
\noindent \textsc{Exercise} 1.2.v. (map that is monic and epi but not an isomorphism)
\begin{solution}
    We denote the inclusion $\Z \to \Q$ by $\iota$. As an injection, $\iota$ is monic. We let $h,k: \Q \rightrightarrows A$ be ring maps such that $h \iota = k \iota$. In other words, for every $n \in \Z$ we have $h(\frac{n}{1}) = k(\frac{n}{1})$. What about fractions of the form $\frac{1}{n}$? We must have $h(\frac{1}{n}) = k(\frac{1}{n})$, as
    \[
    1_A = h(\frac{n}{1})h(\frac{1}{n}) = k(\frac{n}{1})h(\frac{1}{n})
    \]
    and multiplicative inverses in a ring are unique. It follows that $h$ and $k$ agree on all elements of $\Q$ and thus denote the same map. Therefore $\iota$ is epi. Of course, $\Q$ and $\Z$ are not isomorphic in $\cat{Ring}$.
\end{solution}
\noindent \textsc{Exercise} 1.2.vi. 
\begin{solution}
    Suppose that $f: x \to y$ is mono and split epi. So, there exists a section $s: y \to x$ such that $fs = 1_y$, and we must show that $sf = 1_x$. This immediately follows from $f$ being epi, since 
    \[
    fsf = 1_y f = f 1_x 
    \]
    Dually, if $f: y \to x$ is instead split mono and epi, then $f$ is also an isomorphism.
\end{solution}
\noindent \textsc{Exercise} 1.2.vii. \begin{solution}
An object $S \in \cat{P}$ is the \textit{supremum} of a subcollection of objects $A \in \cat{P}$ if 
\begin{enumerate}
\item there exists a morphism $A \to S$ for all $A$ in the subcollection 
\item if, for some object $A$ in the subcollection, there exists a morphism $A \to S'$ for some object $S' \in \cat{P}$, then there exists a morphism $S \to S'$. 
\end{enumerate}
The dual of this statement gives the definition for the \textit{infimum}, which can be equivalently defined as the supremum in $\cat{P}^{\text{op}}$.
\end{solution}
\noindent \textsc{Exercise} 1.3.i. \begin{solution}
    A group homomorphism. More precisely, given a group homomorphism $f: G 
    \to H$, if we regard $G, H$ as one-object categories then the axioms for functoriality correspond with the group homomorphism axioms. 
\end{solution}
\noindent \textsc{Exercise} 1.3.ii. \begin{solution}
A monotone function, similar to exercise 1.3.i. Since "monotone" has different definitions in mathematics, I mean a function $f: P \to Q$ is $\textit{monotone}$ if given $x \leq y$ in $P$, then $f(x) \leq f(y)$ in $Q$. 
\end{solution}
\noindent \textsc{Exercise} 1.3.iii. (the image of a functor is not necessarily a category)
\begin{solution}
    This is a spin on the classic solution to this exercise, featuring more algebra (since algebra is my favourite kind of math). It relies on the same point: composites that exist in the codomain category may not exist in the domain category. Consider the following diagram, under the forgetful functor $\cat{Ring} \to \cat{Ab}$. We use the fact that the abelian group $\Z/p \times \Z/p$ admits multiple nonisomorphic ring structures (here $p$ is a prime).
    \begin{equation}
    \begin{tikzcd}
        \mathbb{F}_{p^2} \arrow[r, "\text{id}"] &\mathbb{F}_{p^2} \\
        &\mathbb{F}_p \times \mathbb{F}_p \arrow[r, "\text{id}"] &\mathbb{F}_p \times \mathbb{F}_p \\
        &\Downarrow \\
        \Z / p \times \Z / p \arrow[r, "\text{id}"] &\Z / p \times \Z / p \arrow[r, "\text{id}"] &\Z/p \times \Z/p
    \end{tikzcd}
    \end{equation}As a field, any maps from $\mathbb{F}_{p^2}$ are necessarily injective. Since $\mathbb{F}_p \times \mathbb{F}_p$ and $\mathbb{F}_{p^2}$ are both finite of cardinality $p^2$, such a map would be an isomorphism, which is not possible since $\mathbb{F}_p \times \mathbb{F}_p$ is not isomorphic to $\mathbb{F}_{p^2}$. \\ \\
Thus, there is no ring map $\mathbb{F}_{p^2} \to \mathbb{F}_p \times \mathbb{F}_p$, so the composite in $\cat{Ab}$ cannot exist back in $\cat{Ring}$. \\ \\
    As an extra fact, unrelated to the problem statement but interesting anyways, there are abelian groups which do not admit a (nontrivial) ring structure, such as $\Q / \Z$, or $\Z(p^{\infty}) := \Z[1/p] / \Z $. In other words, the there are objects of $\cat{Ab}$ that are not in the image of the forgetful functor $\cat{Ring} \to \cat{Ab}$.
\end{solution}
\noindent \textsc{Exercise} 1.3.iv. Verify that the constructions introduced in Definition 1.3.11 are functorial. \\ \\
\textsc{Exercise} 1.3.v. What is the difference between a functor $\textsf{C}^{\text{op}} \to \textsf{D}$ and a functor $\textsf{C} \to \textsf{D}^{\text{op}}$? What is the difference between a functor $\textsf{C} \to \textsf{D}$ and a functor $\textsf{C}^{\text{op}} \to \textsf{D}^{\text{op}}$? \\ \\
\textsc{Exercise} 1.3.vi. (Comma category definition, dom / cod functors)
\begin{solution}
    For an object $(d, e, f)$ in $F \downarrow G$, her identity morphism is $(1_{Fd}, 1_{Ge})$. The square 
    \[
    \begin{tikzcd}
        Fd \arrow[d, "1_{Fd}"] \arrow[r, "f"] &Ge \arrow[d, "1_{Ge}"] \\
        Fd \arrow[r, "f"] &Ge \\
    \end{tikzcd}
    \]
    commutes, and 
    \[
    (1_{Fd}, 1_{Ge}) \circ (h, k) = (h, k) = (h, k) \circ (1_{Fd'}, 1_{Ge'}).
    \]
    We hope there is no confusion about how composition is defined. Similarly to how identities were shown, $F \downarrow G$ inherits associativity from associativity in $\cat{C}, \cat{D}, \cat{E}$, and functoriality of $F$ and $G$ is necessary. So, $F \downarrow G$ is a category, and we define dom and cod by
    \begin{align*}
    &\text{dom}(d, e, f) = d \hspace{1cm} \text{cod}(d, e, f) = e\\ 
    &\text{dom}(h, k) = h  \hspace{1.433cm} \text{cod}(h,k) = k
    \end{align*}
    
        

\end{solution}    
\textsc{Exercise} 1.3.vii. Define functors to construct the slice categories in $c/C$ and $C/c$ of Exercise 1.1.iii as special cases of comma categories constructed in Exercise 1.3.vi. What are the projection functors? \\ \\
% do after 1.3.vi
\textsc{Exercise} 1.3.viii. Lemma 1.3.8 shows that functors preserve isomorphisms. Find an example to demonstrate that functors need not \textbf{reflect isomorphisms}: that is, find a functor $F: \textsf{C} \to \textsf{D}$ and a morphism $f$ in $\textsf{C}$ so that $Ff$ is an isomorphism in $\textsf{D}$ but $f$ is not an isomorphism in $\textsf{C}$. \\ \\
\textsc{Exercise} 1.3.ix. For any group $G$, we may define other groups:
\begin{itemize}
    \item the \textbf{center} $Z(G) = \{ h \in G \: | \: hg=gh \; \forall g \in G \}$, a subgroup of $G$. 
    \item the \textbf{commutator subgroup} $C(G)$, the subgroup of $G$ generated by elements $ghg^{-1}h^{-1}$ for any $g,h \in G$, and
    \item the \textbf{automorphism group} $\text{Aut}(G)$, the group of isomorphisms $\phi: G \to G$ in \cat{Group}.
\end{itemize}
Trivially, all three constructions define a functor from the discrete category of groups (with only identity morphisms) to \cat{Group}. Are these constructions functorial in 
\begin{itemize}
    \item the isomorphisms of groups? That is, do they extend to functors $\cat{Group}_{\text{iso}} \to \cat{Group}$? 
    \item the epimorphisms of groups? That is, do they extend to functors $\cat{Group}_{\text{epi}} \to \cat{Group}$? 
    \item all homomorphisms of groups? That is, do they extend to functors $\cat{Group} \to \cat{Group}$?
\end{itemize}
%seems confusing on first readthrough. from the wording, I expect it extends the first two but not the last.
\textsc{Exercise} 1.3.x. Show that the construction of the set of conjugacy classes of elements of a goup is functorial, defining a functor $\text{Conj}: \cat{Group} \to \cat{Set}$. Conclude that any pair of groups whose sets of conjugacy classes of elements have differing cardinalities cannot be isomorphic.
    
 \end{document}
