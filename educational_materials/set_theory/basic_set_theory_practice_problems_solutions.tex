
\documentclass{article}

\usepackage[margin=1.5in]{geometry} % Please keep the margins at 1.5 so that there is space for grader comments.
\usepackage{amsmath,amsthm,amssymb,hyperref}
\usepackage{tikz-cd}

\newcommand{\R}{\mathbb{R}}  
\newcommand{\Z}{\mathbb{Z}}
\newcommand{\N}{\mathbb{N}}
\newcommand{\Q}{\mathbb{Q}}
\newcommand{\C}{\mathbb{C}}
\newcommand\myeq{\stackrel{\mathclap{\normalfont\mbox{def}}}{=}}
\newcommand{\Aut}{\text{Aut}} 
\newcommand{\Inn}{\text{Inn}}
\newcommand{\End}{\text{End}}
\newcommand{\GL}{\text{GL}}
\newcommand{\PGL}{\text{PGL}}
\newcommand{\im}{\text{im}\:}
\newcommand{\Hom}{\text{Hom}}
\newcommand{\span}{\text{span}}

\newenvironment{theorem}[2][Theorem]{\begin{trivlist}
\item[\hskip \labelsep {\bfseries #1}\hskip \labelsep {\bfseries #2.}]}{\end{trivlist}}
\newenvironment{lemma}[2][Lemma]{\begin{trivlist}
\item[\hskip \labelsep {\bfseries #1}\hskip \labelsep {\bfseries #2.}]}{\end{trivlist}}
\newenvironment{claim}[2][Claim]{\begin{trivlist}
\item[\hskip \labelsep {\bfseries #1}\hskip \labelsep {\bfseries #2.}]}{\end{trivlist}}
\newenvironment{problem}[2][Problem]{\begin{trivlist}
\item[\hskip \labelsep {\bfseries #1}\hskip \labelsep {\bfseries #2.}]}{\end{trivlist}}
\newenvironment{proposition}[2][Proposition]{\begin{trivlist}
\item[\hskip \labelsep {\bfseries #1}\hskip \labelsep {\bfseries #2.}]}{\end{trivlist}}
\newenvironment{corollary}[2][Corollary]{\begin{trivlist}
\item[\hskip \labelsep {\bfseries #1}\hskip \labelsep {\bfseries #2.}]}{\end{trivlist}}

\newenvironment{solution}{\begin{proof}[Solution]}{\end{proof}}

\begin{document}

\noindent SOLUTIONS: \\ \\
1. Listed union, intersection. \\ \\
(a) $\{0,1,2\}, \{1\}$
\\ \\
(b) $\{-1,0,1,2\}$, $\varnothing$
\\ \\
(c) $\R, \Q$
\\ \\
(d) $\R, \{0\}$\\ \\
(e) $\R - \{0\}$ (or in set builder notation: $\{x\in \R: x \neq 0\}$), $\varnothing$.
\\ \\
(f) $\Z$, $\varnothing$
\\ \\
(g) $\Q, \Z$ \\ \\
2. (b), (e), and (f) are disjoint. \\ \\
3. For (b) $\varnothing \subset \{-1,0,1,2\}$, although if you missed this one don't worry. We define the empty set as a subset of every set. It works with the container metaphor, because every container full of stuff still has a container (every set has the empty set as a subset). For (c) $\Q \subset \R$, and for (g) $\Z \subset \Q$. For the rest, neither set is a subset of the other. \\ \\
4. 
(a) $\{2,3,4,5,6,7\}$. \\ \\
(b) $\{n \in \Z: n \text{ is odd}\}$.
\\ \\
(c) $\{x\in \R: x < 0 \text{ or } x > 1$\}. Alternately we could write it as $(-\infty, 0) \cup (1, \infty)$.\\ \\
(d) $\{x\in \R: x \leq 0 \text{ or } x \geq 1$\}. Alternately we could write it as $(-\infty, 0] \cup [1, \infty)$.\\ \\
5. 
(a) 4 \\ \\
(b) 1 \\ \\
(c) $\infty$ \\ \\
(d) 0. This is because this is actually the empty set. \\ \\
6. The sets in (d) are $\{x\in \R: x \leq 0\}$ and $\{x\in \R: x \geq 0\}$, respectively. The sets in (e) are $\{ x \in \R: x < 0\}$ and $\{ x \in \R: x > 0\}$, respectively. 

\end{document}


