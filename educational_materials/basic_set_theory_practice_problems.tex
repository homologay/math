
\documentclass{article}

\usepackage[margin=1.5in]{geometry} % Please keep the margins at 1.5 so that there is space for grader comments.
\usepackage{amsmath,amsthm,amssymb,hyperref}
\usepackage{tikz-cd}

\newcommand{\R}{\mathbb{R}}  
\newcommand{\Z}{\mathbb{Z}}
\newcommand{\N}{\mathbb{N}}
\newcommand{\Q}{\mathbb{Q}}
\newcommand{\C}{\mathbb{C}}
\newcommand\myeq{\stackrel{\mathclap{\normalfont\mbox{def}}}{=}}
\newcommand{\Aut}{\text{Aut}} 
\newcommand{\Inn}{\text{Inn}}
\newcommand{\End}{\text{End}}
\newcommand{\GL}{\text{GL}}
\newcommand{\PGL}{\text{PGL}}
\newcommand{\im}{\text{im}\:}
\newcommand{\Hom}{\text{Hom}}
\newcommand{\span}{\text{span}}

\newenvironment{theorem}[2][Theorem]{\begin{trivlist}
\item[\hskip \labelsep {\bfseries #1}\hskip \labelsep {\bfseries #2.}]}{\end{trivlist}}
\newenvironment{lemma}[2][Lemma]{\begin{trivlist}
\item[\hskip \labelsep {\bfseries #1}\hskip \labelsep {\bfseries #2.}]}{\end{trivlist}}
\newenvironment{claim}[2][Claim]{\begin{trivlist}
\item[\hskip \labelsep {\bfseries #1}\hskip \labelsep {\bfseries #2.}]}{\end{trivlist}}
\newenvironment{problem}[2][Problem]{\begin{trivlist}
\item[\hskip \labelsep {\bfseries #1}\hskip \labelsep {\bfseries #2.}]}{\end{trivlist}}
\newenvironment{proposition}[2][Proposition]{\begin{trivlist}
\item[\hskip \labelsep {\bfseries #1}\hskip \labelsep {\bfseries #2.}]}{\end{trivlist}}
\newenvironment{corollary}[2][Corollary]{\begin{trivlist}
\item[\hskip \labelsep {\bfseries #1}\hskip \labelsep {\bfseries #2.}]}{\end{trivlist}}

\newenvironment{solution}{\begin{proof}[Solution]}{\end{proof}}

\begin{document}

\noindent 1. Find the union and intersection of each of the following pairs of sets: \\ \\
(a) $\{1,2\}$ and $\{1,0\}$
\\ \\
(b) $\varnothing$ (empty set) and $\{-1,0,1,2\}$
\\ \\
(c) $\R$ (real numbers) and $\Q$ (rational numbers)
\\ \\
(d) $(-\infty, 0]$ (all the negative real numbers and 0), and $[0,\infty)$ (all the positive real numbers and 0) \\ \\
(e) $(-\infty,0)$ and $(0, \infty)$. Note that the only difference between these ones and the previous sets is these ones don't include 0. 
\\ \\
(f) $\{n \in \Z: n \text{ is odd} \}$ and $\{n \in \Z: n \text{ is even}\}$. In case it's useful the formal definition of an odd number is a number that can be written $2k+1$ for some integer $k$, and an even number can be written $2k$ (could be different $k$).
\\ \\
(g) $\Q$ (rational numbers) and $\Z$ (integers). \\ \\
2. Which of the pairs of sets in problem 1 are disjoint? \\ \\
3. For some of the pairs of sets in problem 1, one of the sets is a subset of the other one. Which pairs of sets have this property, and for those pairs, which set is a subset of the other one? \textit{For example, in part (c) $\Q$ is a subset of $\R$, since every rational number is also a real number. We write this $\Q \subset \R$. In (a), neither set is a subset of the other one because $2$ is in the first set but not the second, and $0$ is in the second set but not the first. } \\ \\
4. Find the complement of the following sets: \\ \\
(a) $\{0,1\}$, the universe set is $\{0,1,2,3,4,5,6,7\}$. \\ \\
(b) $\{ n \in \Z : n \text{ is even}\}$, the universe set is $\Z$.
\\ \\
(c) $[0,1]$ (the interval from 0 to 1 including the endpoints), the universe set is $\R$. \\ \\
(d) $(0,1)$ (the interval from 0 to 1 not including the endpoints), the universe set is $\R$. \\ \\
5. What is the cardinality of the following sets? It will either be a positive integer (could be 0 but only if it's the empty set) or infinite ($\infty$). \\ \\
(a) $\{0,2,4,6\}$ \\ \\
(b) $\{0\}$ \\ \\
(c) $\R$ \\ \\
(d) $\Z \cap (\R - \Q)$ \\ \\
6. Write the sets in (d) and (e) of problem 1 in set-builder notation: \\ \\
$\{ x \in \{\text{bigger set} \} : \text{ condition on }x \}$  \\ \\
The sets in part (f) of problem 1 are the odd integers and even integers (respectively) written in set builder notation, for an example. 

\end{document}


